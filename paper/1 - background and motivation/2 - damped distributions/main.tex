\subsection{Damped Distributions}\label{damped distributions}
    Another common effect to consider is Alfvénic damping, wherein high-energy particles with velocities on the scale of the Alfvén velocity $v_{A} = \frac{B}{\sqrt{\mu_{0}\rho}}$ lose energy due to resonance with the Alfvén waves, causing velocity distributions with sharper cut-offs than the $\calO\left[\exp\left(- {\rm const.}v^{2}\right)\right]$ tails in Maxwellians, shifted Maxwellians, bi-Maxwellians etc. A general way to consider such ``damped distributions'' is presented here.
    
    Suppose, before damping, the background has some classic distribution, e.g. a shifted Maxwellian, with distribution function $f_{s0}^{\rm pre}\left[\psi; v_{\parallel}^{\rm pre}, \mu^{\rm pre}\right]$ at a given position. Consider this velocity being modified by some ``damping function'', $(v_{\parallel}^{\rm pre}, \mu^{\rm pre}) \mapsto (v_{\parallel}, \mu)$ that reduces the velocity of the high-velocity particles, but leaves the low-velocity particles unchanged. Denote this as:
    \begin{align}
        v_{\parallel}(v_{\parallel}^{\rm pre}, \mu^{\rm pre})  &&
        \mu_{\parallel}(v_{\parallel}^{\rm pre}, \mu^{\rm pre})
    \end{align}
    Similarly, this has the inverse:
    \begin{align}
        v_{\parallel}^{\rm pre}(v_{\parallel}, \mu)  &&
        \mu_{\parallel}^{\rm pre}(v_{\parallel}, \mu)
    \end{align}
    A reasonable condition on this damping function is that it leaves the velocity of low-energy particles unchanged, namely:
    \begin{align}
        \partial_{v_{\parallel}^{\rm pre}}v_{\parallel}(0, 0)  =  1  &&
        \partial_{v_{\parallel}^{\rm pre}}\mu(0, 0)  =  0  \\
        \partial_{\mu^{\rm pre}}v_{\parallel}(0, 0)  =  0  &&
        \partial_{\mu^{\rm pre}}\mu(0, 0)  =  1
    \end{align}
    Similarly, for the inverse:
    \begin{align}
        \partial_{v_{\parallel}}v_{\parallel}^{\rm pre}(0, 0)  =  1  &&
        \partial_{v_{\parallel}}\mu^{\rm pre}(0, 0)  =  0  \\
        \partial_{\mu}v_{\parallel}^{\rm pre}(0, 0)  =  0  &&
        \partial_{\mu}\mu^{\rm pre}(0, 0)  =  1
    \end{align}
    The distribution function $f_{s0}$ for the damped distribution will then evaluate as
    \begin{equation}\label{damped distribution function}
        f_{s0}[\psi; v_{\parallel}, \mu]  =  J(v_{\parallel}, \mu)f_{s0}^{\rm pre}\left[\psi; v_{\parallel}^{\rm pre}(v_{\parallel}, \mu), \mu^{\rm pre}(v_{\parallel}, \mu)\right]
    \end{equation}
    where $J$ is the Jacobian
    \begin{equation}\label{Jacobian}
        J(v_{\parallel}, \mu)  :=  \partial_{v_{\parallel}}v_{\parallel}^{\rm pre}(v_{\parallel}, \mu)\partial_{\mu}\mu^{\rm pre}(v_{\parallel}, \mu) - \partial_{\mu}v_{\parallel}^{\rm pre}(v_{\parallel}, \mu)\partial_{v_{\parallel}}\mu^{\rm pre}(v_{\parallel}, \mu)
    \end{equation}
    As such, to create a damped velocity distribution, only 3 ``ingredients'' are needed:
    \begin{itemize}
        \item  $f_{s0}^{\rm pre}\left[\psi; v_{\parallel}^{\rm pre}, \mu^{\rm pre}\right]$, the velocity distribution \emph{before damping}.
        \item  $v_{\parallel}^{\rm pre}(v_{\parallel}, \mu)$, the parallel velocity a damped particle had \emph{before damping}.
        \item  $\mu^{\rm pre}(v_{\parallel}, \mu)$, the magnetic moment a damped particle had \emph{before damping}.
    \end{itemize}

    \begin{example}{\emph{(Uniform damping)}}
        Consider a damping function wherein a particle's pitch angle remains unaffected by the damping, but the speed $v = \sqrt{v_{\parallel}^{2} + \frac{2B}{m_{s}}\mu}$ is damped uniformly among all pitch angles through some function $v^{\rm pre}(v)$, such that:
        \begin{align}
           v_{\parallel}^{\rm pre}(v_{\parallel}, \mu)  &=  \frac{v^{\rm pre}(v)}{v}v_{\parallel}  \\
           \mu^{\rm pre}(v_{\parallel}, \mu)  &=  \frac{v^{\rm pre}(v)}{v}\mu
        \end{align}
        Such a $v^{\rm pre}(v)$ would need to:
        \begin{itemize}
            \item  satisfy the low-velocity condition $\partial_{v}v^{\rm pre}(0) = 1$.
            \item  go to $\infty$ much faster than $v$ for $v > v_{A}$ (or whatever the velocity scale on which the damping takes effects may be).
        \end{itemize}
        Such a function may for example take the form
        \begin{equation}
            v^{\rm pre}(v)  :=  \left[1 + \left(\frac{v}{v_{A}}\right)^{r}\right]v
        \end{equation}
    \end{example}
    