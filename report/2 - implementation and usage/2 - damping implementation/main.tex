\subsection{Damping Implementation}\label{damping implementation}
    ``Damped distribution functions'' are incorporated in this work into {\tt GENE} similarly how one would any other distribution function: with a new {\tt dist\_eq\_*\_damped\_t} class (in a {\tt dist\_eq\_*\_damped\_m} module) extending {\tt dist\_eq\_t}. Whereas a traditional background distribution needs only its distribution function, $f_{s0}$ (or more accurately its distribution function's parameters, $n_{s}$, $u_{\parallel s}$, $T_{s}$, etc.) to be defined, a damped distribution requires (as highlighted in subsection \ref{damped distributions}) 3 ``ingredients'':
    \begin{itemize}
        \item  $f_{s0}^{\rm pre}\left[\psi; v_{\parallel}^{\rm pre}, \mu^{\rm pre}\right]$ (or more accurately the parameters, $n_{s}$, $u_{\parallel s}$, $T_{s}$, etc. of) the velocity distribution \emph{before damping}.
        \item  $v_{\parallel}^{\rm pre}(v_{\parallel}, \mu)$, the parallel velocity a damped particle had \emph{before damping}.
        \item  $\mu^{\rm pre}(v_{\parallel}, \mu)$, the magnetic moment a damped particle had \emph{before damping}.
    \end{itemize}
    Within the implementation, therefore, a class/module for a \emph{damped} distribution looks almost identical to one for a \emph{non-damped} distribution, except for its definition containing a new variable that carries information on the damping function, $(v_{\parallel}, \mu)  \mapsto  (v_{\parallel}^{\rm pre}, \mu^{\rm pre})$. This is done through:
    \begin{enumerate}
        \item  General implementation of the new class, {\tt damping\_t} (in the {\tt damping\_base\_m} module), the class of damping functions, with space all the relevant information on the damping function.
        \item  User implementation of a new class, {\tt damping\_*\_t} (in a {\tt damping\_*\_m} module), extending {\tt damping\_t}, for each specific damping function—say, \emph{uniform} damping, as defined in subsection \ref{damped distributions}—the user would like to consider, defining the damping function $(v_{\parallel}, \mu)  \mapsto  (v_{\parallel}^{\rm pre}, \mu^{\rm pre})$ in consideration.
        \item  User implementation of a new subclass {\tt dist\_eq\_*\_damped\_t} (in a {\tt dist\_eq\_*\_damped\_m} module), extending {\tt dist\_eq\_t}, just as one might do already to implement a new distribution, except this time containing (and initializing) a new variable {\tt dist\_eq\_*\_damped\%damp} of type {\tt damping\_*\_t} containing information on the damping function in question.
    \end{enumerate}
    This new class structure is highlighted in {\color{red!90} red} in Figure \ref{class structure after damping}.
    
    \input{2 - implementation and usage/2 - damping implementation/tikz/class diagram 2.tex}

    As for what information should be contained in the {\tt damping\_t} class, consider what information is needed about the damping function to evaluate all the procedures in Figure \ref{distribution procedures}. Since both {\tt get\_*D\_dvdm} and the terms in the last two sections can be written\footnote{if they cannot be evaluated directly} in terms of those in the first section, and the only component of {\tt get\_edr} that changes between distribution function classes is that which is featured in {\tt get\_edr\_prefactor}, the only 3 terms that actually need evaluating to implement a new distribution function class are simply the 3 derivatives of $f_{s0}$ in $v_{\parallel}$, $\mu$, $\psi$:
    \begin{center}\begin{tabular}{ c | c }
        Procedure  &  Functional  \\
        \hline\hline
        {\tt get\_*D\_dv}  &  $\partial_{v_{\parallel}}f_{s0}$  \\
        \hline
        {\tt get\_*D\_dm}  &  $\partial_{\mu}f_{s0}$  \\
        \hline\hline
        {\tt get\_edr\_prefactor}  &  $\partial_{\psi}f_{s0}$ (without $B$ derivative?)
    \end{tabular}\end{center}
    Recall then equation (\ref{damped distribution function}) for $f_{s0}$
    \begin{equation*}
        f_{s0}[\psi; v_{\parallel}, \mu]  =  J(v_{\parallel}, \mu)f_{s0}^{\rm pre}\left[\psi; v_{\parallel}^{\rm pre}(v_{\parallel}, \mu), \mu^{\rm pre}(v_{\parallel}, \mu)\right]
    \end{equation*}
    alongside equation (\ref{Jacobian}) for $J$
    \begin{equation*}
        J(v_{\parallel}, \mu)  :=  \partial_{v_{\parallel}}v_{\parallel}^{\rm pre}(v_{\parallel}, \mu)\partial_{\mu}\mu^{\rm pre}(v_{\parallel}, \mu) - \partial_{\mu}v_{\parallel}^{\rm pre}(v_{\parallel}, \mu)\partial_{v_{\parallel}}\mu^{\rm pre}(v_{\parallel}, \mu)
    \end{equation*}
    The 3 relevant derivatives in $f_{s0}$ then evaluate as:
    \begin{align}
        \partial_{v_{\parallel}}f_{s0}
            &=  \partial_{v_{\parallel}}J\cdot f_{s0}^{\rm pre}
            + J\cdot\left(\partial_{v_{\parallel}^{\rm pre}}f_{s0}^{\rm pre}\cdot\partial_{v_{\parallel}}v_{\parallel}^{\rm pre}
            + \partial_{\mu^{\rm pre}}f_{s0}^{\rm pre}\cdot\partial_{v_{\parallel}}\mu^{\rm pre}\right)  \\
        \partial_{\mu}f_{s0}
            &=  \partial_{\mu}J\cdot f_{s0}^{\rm pre}
            + J\cdot\left(\partial_{v_{\parallel}^{\rm pre}}f_{s0}^{\rm pre}\cdot\partial_{\mu}v_{\parallel}^{\rm pre}
            + \partial_{\mu^{\rm pre}}f_{s0}^{\rm pre}\cdot\partial_{\mu}\mu^{\rm pre}\right)  \\
        \partial_{\psi}f_{s0}
            &=  J\cdot\partial_{\psi}f_{s0}^{\rm pre}
    \end{align}
    alongside the derivatives in $J$:
    \begin{align}
        \partial_{v_{\parallel}}J
            &=  \left(\partial_{v_{\parallel}}^{2}v_{\parallel}^{\rm pre}\cdot\partial_{\mu}\mu^{\rm pre} + \partial_{v_{\parallel}}v_{\parallel}^{\rm pre}\cdot\partial_{v_{\parallel}}\partial_{\mu}\mu^{\rm pre}\right)
            - \left(\partial_{v_{\parallel}}\partial_{\mu}v_{\parallel}^{\rm pre}\cdot\partial_{v_{\parallel}}\mu^{\rm pre} + \partial_{\mu}v_{\parallel}^{\rm pre}\cdot\partial_{v_{\parallel}}^{2}\mu^{\rm pre}\right)  \\
        \partial_{v_{\parallel}}J
            &=  \left(\partial_{v_{\parallel}}\partial_{\mu}v_{\parallel}^{\rm pre}\cdot\partial_{\mu}\mu^{\rm pre} + \partial_{v_{\parallel}}v_{\parallel}^{\rm pre}\cdot\partial_{\mu}^{2}\mu^{\rm pre}\right)
            - \left(\partial_{\mu}^{2}v_{\parallel}^{\rm pre}\cdot\partial_{v_{\parallel}}\mu^{\rm pre} + \partial_{\mu}v_{\parallel}^{\rm pre}\cdot\partial_{v_{\parallel}}\partial_{\mu}\mu^{\rm pre}\right)
    \end{align}
    A new class of damping function {\tt damping\_*\_t} should therefore specify, not only $v_{\parallel}^{\rm pre}$, $\mu^{\rm pre}$, but their derivatives up to 2nd order. The evaluation of $J$ and its derivatives is identical for all damping functions, and so is implemented in general as an internal procedure contained in {\tt damping\_t}.